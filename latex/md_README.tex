S\+A\+N\+D\+A\+L2 is a S\+D\+L2 wrapper which purpose is to make objects managment and graphic display easier.

\subsection*{About the author}

I am a French guy and I am studying computer science engineering. I made this wrapper because I found boresome to have to create a display function for each menu\textquotesingle{}s page, but I ended up going a little more further in it. I hope you will enjoy using it, if you have any comment or advice, do not feel shy and tell me !~\newline
 I will really appreciate it.

\subsection*{Structures}

\subsubsection*{I. \hyperlink{HitBox_8h}{Hit\+Box.\+h}}


\begin{DoxyEnumerate}
\item \hyperlink{structLineSDL2}{Line\+S\+D\+L2} \+: A clickable zone delimited by a line (only one side of the line is clickable).
\item \hyperlink{structCircleSDL2}{Circle\+S\+D\+L2} \+: An elliptic shaped clickable zone.
\item \hyperlink{structHitBox}{Hit\+Box} \+: A clickable zone made of a collection of lines and ellipses. This is one of the only structure you\textquotesingle{}ll be manipulating (even though not directly accessing to its values).
\item \hyperlink{structListHitBox}{List\+Hit\+Box} \+: A clickable zone defined by a collection of clickable \hyperlink{structHitBox}{Hit\+Box} and blocking \hyperlink{structHitBox}{Hit\+Box}.
\end{DoxyEnumerate}

\subsubsection*{II. \hyperlink{DisplayCode_8h}{Display\+Code.\+h}}


\begin{DoxyEnumerate}
\item \hyperlink{structDisplayCode}{Display\+Code} \+: A package of information about a display code. It contains the display code, the plan and a flag to tell whether or not the object should be displaied.
\item \hyperlink{structListDisplayCode}{List\+Display\+Code} \+: A list of display codes.
\end{DoxyEnumerate}

\subsubsection*{I\+II. \hyperlink{FontSDL2_8h}{Font\+S\+D\+L2.\+h}}


\begin{DoxyEnumerate}
\item \hyperlink{structFontSDL2}{Font\+S\+D\+L2} \+: A package of information about a text. It contains its font, its text, its color and its S\+D\+L2 texture.
\end{DoxyEnumerate}

\subsubsection*{IV. Fenetre\+S\+D\+L2.\+h}


\begin{DoxyEnumerate}
\item Fenetre\+S\+D\+L2 \+: Representation of a window. It contains all informations (such as the dimension, the display code, ...) about a window.
\item List\+Fenetre\+S\+D\+L2 \+: A list of Fenetre\+S\+D\+L2.
\end{DoxyEnumerate}

\subsubsection*{V. \hyperlink{ElementSDL2_8h}{Element\+S\+D\+L2.\+h}}


\begin{DoxyEnumerate}
\item \hyperlink{structElementSDL2}{Element\+S\+D\+L2} \+: A package of information about objects. It contains its coordinates (top left corner), its size, its color and so on. This is one of the only structure you\textquotesingle{}ll be manipulating (even though not directly accessing to its values).
\item \hyperlink{structPtrElementSDL2}{Ptr\+Element\+S\+D\+L2} \+: A structure mean to store an \hyperlink{structElementSDL2}{Element\+S\+D\+L2}\textquotesingle{}s pointer
\item \hyperlink{structListPtrElementSDL2}{List\+Ptr\+Element\+S\+D\+L2} \+: A list of \hyperlink{structPtrElementSDL2}{Ptr\+Element\+S\+D\+L2}. Either used as a list of elements with a common plan or so that an element can change another.
\item \hyperlink{structListDCElementSDL2}{List\+D\+C\+Element\+S\+D\+L2} \+: A list of lists of elements with a common plan, every lists in this list has a common display code.
\item \hyperlink{structListElementSDL2}{List\+Element\+S\+D\+L2} \+: A list of \hyperlink{structListDCElementSDL2}{List\+D\+C\+Element\+S\+D\+L2}.
\end{DoxyEnumerate}

\subsection*{Functionnality}

\subsubsection*{I. To begin with}

To begin with, you\textquotesingle{}ll have to initialise S\+D\+L2 with init\+All\+S\+D\+L2 for image, ttf and S\+D\+L2, or one of the specific function in \hyperlink{SANDAL2_8h}{S\+A\+N\+D\+A\+L2.\+h}. Do not forget to close the S\+D\+L2 with either close\+All\+S\+D\+L2 or one or more functions in \hyperlink{SANDAL2_8h}{S\+A\+N\+D\+A\+L2.\+h}.~\newline
~\newline
 Then you\textquotesingle{}ll want to create a window (or more). For that, use the init\+Fenetre\+S\+D\+L2 function. Then you can had elements in it with all the create functions in \hyperlink{ElementSDL2_8h}{Element\+S\+D\+L2.\+h} like create\+Block or create\+Entry\+Image for example. We will see later what you can do with all those elements more in details.~\newline
~\newline
 For the event management, you can bind functions to elements which will be called in the functions we will speak about right now when specific conditions are met. You will not have to look when you touch an elements or what function to call when a key is pressed. For that, use the functions like click\+Fenetre\+S\+D\+L2 or key\+Pressed\+Fenetre\+S\+D\+L2. When you want to update all your elements and the current window, use update\+Fenetre\+S\+D\+L2. For displaying, use display\+Fenetre\+S\+D\+L2. If you want to do that for every single window, use the \textquotesingle{}all\textquotesingle{} version like display\+All\+Fenetre\+S\+D\+L2 or unclick\+All\+Fenetre\+S\+D\+L2. Then again, do not forget to close your windows with close\+Fenetre\+S\+D\+L2 or close\+All\+Fenetre\+S\+D\+L2. Those functions are in \hyperlink{SANDAL2_8h}{S\+A\+N\+D\+A\+L2.\+h}.~\newline
~\newline
 If you want to iterate through all the windows, you can use init\+Iterator\+Fenetre\+S\+D\+L2 and next\+Fenetre\+S\+D\+L2. You can also get and set informations about the window with all the functions in Fenetre\+S\+D\+L2.\+h.~\newline


\subsubsection*{II. Element manipulations}

An element is kind of an object with lots of display informations. It can have an image, a text or a color (for rectangles). As said higher, they can be created with functions like create\+Block or create\+Button for example. Once created, you can modifie all informations about them. They have the following information \+:
\begin{DoxyItemize}
\item coordinates in the window ;
\item dimensions ;
\item a collection of display code, each display code has a plan ;
\item a collection of clickable zone ;
\item functions bind on it to be called for certain event (they will be listed later) ;
\item a collection of animations ;
\item a collection of elements which can be modified by the parent element ;
\item a rotation speed ;
\item a rotation (the current angle of the element) ;
\item a package of data (void $\ast$);
\item a package of informations if the element is a prompt. ~\newline
~\newline


The functions that can be bind to the element are the following ones \+:
\item action, to be called every update ;
\item key\+Press, to be called when a key is pressed ;
\item key\+Released, to be called when a key is released ;
\item click, to be called if the element is clicked (be careful, there have to be a clickable zone) ;
\item un\+Click, to be called when the user release the click of the mouse on the clickable zone of the element ;
\item un\+Select, to be called when the user click elsewhere or unclick elsewhere ;
\item end\+Sprite, to be called when an animation reach its end, before starting of again. ~\newline
~\newline


You can add elements to a current element so that the parent element will be able to modifie them in one of the function you binded to it. In those function, you can modifie the element itself or other, even removing them. Each animation of an element has a code so that you can switch from one to another. When you want to add a step to the animation, you will have to define its coordinates in the image of the element (which should be a sprite sheet or something the like). Each sprite has a lifespan, which correspond to the number of update call before going to the next. That is if the animation is on \textquotesingle{}automatic\textquotesingle{} mode. The mode can be set with the function set\+Way\+Animation\+Element\+S\+D\+L2. If there are no sprite, the entire image will be displaied. ~\newline
~\newline
 
\end{DoxyItemize}